\documentclass{article}


\usepackage{fontspec}   %加這個就可以設定字體
\usepackage{xeCJK}       %讓中英文字體分開設置
\setCJKmainfont{標楷體} %設定中文為系統上的字型,而英文不去更動,使用原TeX字型
\XeTeXlinebreaklocale "zh"             %這兩行一定要加,中文才能自動換行
\XeTeXlinebreakskip = 0pt plus 1pt     %這兩行一定要加,中文才能自動換行

\begin{document}

	\begin{center}
	  \LARGE{統算EM report - }
	  統計碩一黃三騰
	\end{center}
	
	\begin{center}
	ON THE CONVERGENCE PROPERTIES OF THE EM ALGORITHM\\
	BY C.F.JEFF WU
	\end{center}
	EM algorithm's advantage and disadvantage:\\
    1. the form of the complete-data likelihood function such that M-step exist in closed form.\\
    2. M-step has a standard statistical package that saving programming time.\\
    3. not require large storage space.\\
    4. slow numerical convergence.\\[0.5cm]
    
    
    討論問題:\\
	(i) EM algorithm find a local maximum or stationary value ?\\
    (ii) convergence of the sequence of parameter estimates generated by EM.\\[0.3cm]
    
    In section 2.1: \\
    For a bounded sequence $L(\phi_p)$, $L(\phi_{p+1})\geq L(\phi_p) $ implies that $L(\phi_p)$ converges monotonically to some $L^*$.\\
    show that whether $L^*$ is the global maximum of $L(\phi)$ over $\Omega$. Or, a local maximum.\\
    In general, if the log--likelihood $L$ has several (local or global) maximum and stationary points, convergence of the EM sequence depends on the choice of starting point.\\[0.3cm]
    
    In section 2.2: \\
    The convergence of $L(\phi)$ to $L^*$ in section 2.1 dose not imply the convergence of $\phi_p$ to $\phi^*$. It requires more stringent regularity conditions.\\
     
            
       
    
\end{document}